\chapter{\label{summary}Summary and Conclusions}



\begin{enumerate}
	\item We learned how the PID controller works.
	\item  We studied the PID controllers theoretical explanations.
	\item  We first implement the PID controller using the TCLab in python.
	\item  We observe the changes in how fast or slow the graph reach to its target with changing parameters.
	\item  We learned how multiSim works. 
	\item We implemented the PID controller in multiSim.
	\item  We observe the change in time to reach the target with changing the value of resistance in the circuit in multiSim.
\end{enumerate}

P-I-D  control  and  its  variations  are  commonly  used  in  the  industry.  They  have  so  many applications. Control engineers usually prefer P-I  controllers to control  first order plants.  On the other hand, P-I-D control is vastly used to control two or higher order plants. In almost all cases  fast  transient  response  and  zero  steady  state  error  is  desired  for  a  closed  loop  system.  Usually, these two specifications conflict with each other which makes the design harder. The reason why P-I-D is preferred is that it provides both of these features at the same time.  In  this  recitation,  it  was  aimed  to  explain  how  one  can  successfully  use  P-I-D  controllers  in their prospective projects. Being prospective  control engineers, we feel  lucky  to  give  a  presentation  on  the  P-I-D  subject.  Finally,  we  encourage  prospective control  engineers  to  use  P-I-D  controllers  wherever  necessary,  especially,  when  a  great controller is required.

\setcounter{equation}{0}
\setcounter{table}{0}
\setcounter{figure}{0}
%\baselineskip 24pt


    




\begin{center}
{\large {\bf  ABSTRACT }}
\end{center}  


    
PID controllers are being widely used in industry due to their well-grounded 
established theory, simplicity, maintenance requirements, and ease of retuning online. 
In  the  past  four  decades,  there  are  numerous  papers  dealing  with  tuning  of  PID 
controller. Designing a PID controller to meet gain and phase margin  specification is 
a  well-known  design  technique.  If  the  gain  and  phase  margin  are  not  specified 
carefully then the design may  not be optimum in the sense that could be  large phase 
margin  (more  robust)  that  could  give  better  performance.  This  paper  studies  the 
relationship between ISE performance index, gain margin, phase margin and 
compares  two  tuning  technique,  based  on  these  three  parameters.  These  tuning 
techniques  are  particularly  useful  in  the  context  of  adaptive  control  and  auto-tuning, 
where the control parameters have to be calculated on-line. 
 
In  the  first  part,  basics  of  various  controllers,  their  working  and  importance  of  PID 
controller in reference to a practical system (thermal control system) is discussed. 
 
In  the  latter  part  of  the  work,  exhaustive  study  has  been  done  on  two  different  PID 
controller tuning techniques. A compromise between robustness and tracking 
performance  of  the  system  in  presence  of  time  delay  is  tried  to  achieve.  Results  of 
simulation, graph, plots, indicate the validity of the study. 


test